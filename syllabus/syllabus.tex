\documentclass[11pt]{article}
\usepackage{fullpage}
\parindent0em\parskip1em
%\textheight10in
%\pagestyle{empty}
%\usepackage{cancel} % for strikeout/cancel


\begin{document}
\begin{center}
\large \textsf{\textbf{CAAM 452 $\cdot$ Numerical solution of partial differential equations}\\[0.5em]
Spring 2021 $\cdot$ Rice University}
\end{center}

\hspace*{-4em}
\begin{tabular}[b]{rp{35em}}

Lectures:		& Online, 1:30-2:50pm Tuesday and Thursday. \\[1em]

Objectives: 	& This course covers various numerical methods for solving partial differential equations, with focus on theoretical aspects and practical implementation (using the Julia programming language) of finite difference methods and finite element methods for elliptic, parabolic, and hyperbolic problems.\\ \\
			& Time permitting, aspects of other numerical methods (e.g., finite volumes, discontinuous Galerkin) may also be covered.\\[1em]

Instructors:  	 & Jesse Chan (jesse.chan at rice.edu), Duncan Hall 3023, 713--348--6113\\[1em]

Office Hours:	 & Tuesdays 2:50-4pm or by appointment.\\[1em]
               
Prerequisites:	 & Calculus, linear algebra, basic programming experience \\[1em]                           

Grading:		 & 10\% attendance and class participation, 90\% homework.\\[1em]

Late Policy:	 & Homeworks may be turned in late with advance instructor permission.\\[1em]

Texts:	  	 & \emph{Finite Difference Methods for Ordinary and Partial Differential Equations: Steady-State and Time-Dependent Problems}  by Randall J. LeVeque.\\ \\
			 & \emph{Understanding and Implementing the Finite Element Method}  by Mark S. Gockenbach. \\[1em] 

Supplementary: & \emph{An analysis of the finite element method} by Gilbert Strang and George Fix\\
			 & \emph{Finite Volume Methods for Hyperbolic Problems} by Randall J. LeVeque.\\[1em]
\end{tabular}

\vspace{2em}
\begin{center}
\em Any student with a disability requiring accommodation in this course is encouraged\\
to contact the instructor during the first week of class, and also to contact\\
Disability Support Services in the Ley Student Center.
\end{center}

%\newpage
%\vspace*{-5em}
%\begin{center} 
%   CAAM 652 $\cdot$ Course Outline
%\end{center}
%
%\begin{enumerate}
%\item Matrix Methods for Electrical Systems
%% Matrix-vector products
%\item Matrix Methods for Mechanical Systems
%% Solving linear systems
%\item The Column and Null Spaces
%% When is a system solvable? 
%\item The Fundamental Theorem of Linear Algebra
%% The four fundamental subspaces
%\item Least Squares
%% Data fitting via projections
%\item Matrix Methods for Dynamical Systems
%\item Complex Numbers and Functions
%\item Complex Integration
%\item Back to Dynamical Systems
%\item The Eigenvalue Problem
%\item The Spectral Representation of a Symmetric Matrix
%\item The Singular Value Decomposition (SVD) 
%%\item Matrix Methods for Biochemical Networks
%\end{enumerate}
%
%\vspace*{1em}
%\begin{center}
%   \emph{Tentative} Exam  Topics
%\end{center}
%There will be three timed (three hour), closed-book, take-home exams.\\
%You will have at least a four day window in which to take each exam.
%\begin{enumerate}
%\item[] Exam 1, Chapters 1--4%: February 11 
%\item[] Exam 2, Chapters 5--8%: March 21
%\item[] Exam 3, Chapters 9--13%: final exam period
%\end{enumerate}

\end{document}
